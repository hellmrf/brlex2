\documentclass[calibri]{brlex2}

\begin{document}

\epigrafe{LEI COMPLEMENTAR Nº 95, DE 26 DE FEVEREIRO DE 1998}
\ementa{Dispõe sobre a elaboração, a redação, a alteração e a consolidação das leis, conforme determina o parágrafo único do art. 59 da Constituição Federal, e estabelece normas para a consolidação dos atos normativos que menciona.}
\preambulo[O PRESIDENTE DA REPÚBLICA]{Faço saber que  o Congresso Nacional decreta e eu sanciono a seguinte Lei Complementar:}
\metadata

\art A elaboração, a redação, a alteração e a consolidação das leis obedecerão ao disposto nesta Lei Complementar.

\parun As disposições desta Lei Complementar aplicam-se, ainda, às medidas provisórias e demais atos normativos referidos no art. 59 da Constituição Federal, bem como, no que couber, aos decretos e aos demais atos de regulamentação expedidos por órgãos do Poder Executivo.

\art (VETADO)

\so (VETADO)

\so Na numeração das leis serão observados, ainda, os seguintes critérios:

\inc as emendas à Constituição Federal terão sua numeração iniciada a partir da promulgação da Constituição;

\inc as leis complementares, as leis ordinárias e as leis delegadas terão numeração seqüencial em continuidade às séries iniciadas em 1946.

\capitulo{DAS TÉCNICAS DE ELABORAÇÃO, REDAÇÃO E ALTERAÇÃO DAS LEIS}

\secao{Da Estruturação das Leis}

\art A lei será estruturada em três partes básicas:

\inc parte preliminar, compreendendo a epígrafe, a ementa, o preâmbulo, o enunciado do objeto e a indicação do âmbito de aplicação das disposições normativas;

\inc parte normativa, compreendendo o texto das normas de conteúdo substantivo relacionadas com a matéria regulada;

\inc parte final, compreendendo as disposições pertinentes às medidas necessárias à implementação das normas de conteúdo substantivo, às disposições transitórias, se for o caso, a cláusula de vigência e a cláusula de revogação, quando couber.

\art A epígrafe, grafada em caracteres maiúsculos, propiciará identificação numérica singular à lei e será formada pelo título designativo da espécie normativa, pelo número respectivo e pelo ano de promulgação.

\art A ementa será grafada por meio de caracteres que a realcem e explicitará, de modo conciso e sob a forma de título, o objeto da lei.

\art O preâmbulo indicará o órgão ou instituição competente para a prática do ato e sua base legal.

\tema{Dos temas}

\art O primeiro artigo do texto indicará o objeto da lei e o respectivo âmbito de aplicação, observados os seguintes princípios:

\inc excetuadas as codificações, cada lei tratará de um único objeto;

\inc a lei não conterá matéria estranha a seu objeto ou a este não vinculada por afinidade, pertinência ou conexão;

\inc o âmbito de aplicação da lei será estabelecido de forma tão específica quanto o possibilite o conhecimento técnico ou científico da área respectiva;

\inc o mesmo assunto não poderá ser disciplinado por mais de uma lei, exceto quando a subseqüente se destine a complementar lei considerada básica, vinculando-se a esta por remissão expressa.

\art A vigência da lei será indicada de forma expressa e de modo a contemplar prazo razoável para que dela se tenha amplo conhecimento, reservada a cláusula ``entra em vigor na data de sua publicação'' para as leis de pequena repercussão.

\so A contagem do prazo para entrada em vigor das leis que estabeleçam período de vacância far-se-á com a inclusão da data da publicação e do último dia do prazo, entrando em vigor no dia subseqüente à sua consumação integral.    (Incluído pela Lei Complementar nº 107, de 26.4.2001)

\so As leis que estabeleçam período de vacância deverão utilizar a cláusula ‘esta lei entra em vigor após decorridos XXX dias de sua publicação oficial.     (Incluído pela Lei Complementar nº 107, de 26.4.2001)

\art A cláusula de revogação deverá enumerar, expressamente, as leis ou disposições legais revogadas.    (Redação dada pela Lei Complementar nº 107, de 26.4.2001)

\parun (VETADO)     (Incluído pela Lei Complementar nº 107, de 26.4.2001)

\secao{Da Articulação e da Redação das Leis}

\art Os textos legais serão articulados com observância dos seguintes princípios:

\inc a unidade básica de articulação será o artigo, indicado pela abreviatura ``Art.'', seguida de numeração ordinal até o nono e cardinal a partir deste;

\inc os artigos desdobrar-se-ão em parágrafos ou em incisos; os parágrafos em incisos, os incisos em alíneas e as alíneas em itens;

\inc os parágrafos serão representados pelo sinal gráfico ``§'', seguido de numeração ordinal até o nono e cardinal a partir deste, utilizando-se, quando existente apenas um, a expressão ``parágrafo único'' por extenso;

\inc os incisos serão representados por algarismos romanos, as alíneas por letras minúsculas e os itens por algarismos arábicos;

\inc o agrupamento de artigos poderá constituir Subseções; o de Subseções, a Seção; o de Seções, o Capítulo; o de Capítulos, o Título; o de Títulos, o Livro e o de Livros, a Parte;

\inc os Capítulos, Títulos, Livros e Partes serão grafados em letras maiúsculas e identificados por algarismos romanos, podendo estas últimas desdobrar-se em Parte Geral e Parte Especial ou ser subdivididas em partes expressas em numeral ordinal, por extenso;

\inc as Subseções e Seções serão identificadas em algarismos romanos, grafadas em letras minúsculas e postas em negrito ou caracteres que as coloquem em realce;

\inc a composição prevista no inciso V poderá também compreender agrupamentos em Disposições Preliminares, Gerais, Finais ou Transitórias, conforme necessário.

\art As disposições normativas serão redigidas com clareza, precisão e ordem lógica, observadas, para esse propósito, as seguintes normas:

\inc para a obtenção de clareza:

\alinea usar as palavras e as expressões em seu sentido comum, salvo quando a norma versar sobre assunto técnico, hipótese em que se empregará a nomenclatura própria da área em que se esteja legislando;

\alinea usar frases curtas e concisas;

\alinea construir as orações na ordem direta, evitando preciosismo, neologismo e adjetivações dispensáveis;

\alinea buscar a uniformidade do tempo verbal em todo o texto das normas legais, dando preferência ao tempo presente ou ao futuro simples do presente;

\alinea usar os recursos de pontuação de forma judiciosa, evitando os abusos de caráter estilístico;

\inc para a obtenção de precisão:

\alinea articular a linguagem, técnica ou comum, de modo a ensejar perfeita compreensão do objetivo da lei e a permitir que seu texto evidencie com clareza o conteúdo e o alcance que o legislador pretende dar à norma;

\alinea expressar a idéia, quando repetida no texto, por meio das mesmas palavras, evitando o emprego de sinonímia com propósito meramente estilístico;

\alinea evitar o emprego de expressão ou palavra que confira duplo sentido ao texto;

\alinea escolher termos que tenham o mesmo sentido e significado na maior parte do território nacional, evitando o uso de expressões locais ou regionais;

\alinea usar apenas siglas consagradas pelo uso, observado o princípio de que a primeira referência no texto seja acompanhada de explicitação de seu significado;

\alinea grafar por extenso quaisquer referências a números e percentuais, exceto data, número de lei e nos casos em que houver prejuízo para a compreensão do texto;   (Redação dada pela Lei Complementar nº 107, de 26.4.2001)

\alinea indicar, expressamente o dispositivo objeto de remissão, em vez de usar as expressões ‘anterior’, ‘seguinte’ ou equivalentes;    (Incluída pela Lei Complementar nº 107, de 26.4.2001)

\inc para a obtenção de ordem lógica:

\alinea reunir sob as categorias de agregação - subseção, seção, capítulo, título e livro - apenas as disposições relacionadas com o objeto da lei;

\alinea restringir o conteúdo de cada artigo da lei a um único assunto ou princípio;

\alinea expressar por meio dos parágrafos os aspectos complementares à norma enunciada no caput do artigo e as exceções à regra por este estabelecida;

\alinea promover as discriminações e enumerações por meio dos incisos, alíneas e itens.

\secao{Da Alteração das Leis}

\art A alteração da lei será feita:

\inc mediante reprodução integral em novo texto, quando se tratar de alteração considerável;

\inc mediante revogação parcial;      (Redação dada pela Lei Complementar nº 107, de 26.4.2001)

\inc nos demais casos, por meio de substituição, no próprio texto, do dispositivo alterado, ou acréscimo de dispositivo novo, observadas as seguintes regras:

\alinea revogado;    (Redação dada pela Lei Complementar nº 107, de 26.4.2001)

\alinea é vedada, mesmo quando recomendável, qualquer renumeração de artigos e de unidades superiores ao artigo, referidas no inciso V do art. 10, devendo ser utilizado o mesmo número do artigo ou unidade imediatamente anterior, seguido de letras maiúsculas, em ordem alfabética, tantas quantas forem suficientes para identificar os acréscimos;      (Redação dada pela Lei Complementar nº 107, de 26.4.2001)

\alinea é vedado o aproveitamento do número de dispositivo revogado, vetado, declarado inconstitucional pelo Supremo Tribunal Federal ou de execução suspensa pelo Senado Federal em face de decisão do Supremo Tribunal Federal, devendo a lei alterada manter essa indicação, seguida da expressão ‘revogado’, ‘vetado’, ‘declarado inconstitucional, em controle concentrado, pelo Supremo Tribunal Federal’, ou ‘execução suspensa pelo Senado Federal, na forma do art. 52, X, da Constituição Federal;        (Redação dada pela Lei Complementar nº 107, de 26.4.2001)

\alinea é admissível a reordenação interna das unidades em que se desdobra o artigo, identificando-se o artigo assim modificado por alteração de redação, supressão ou acréscimo com as letras ‘NR’ maiúsculas, entre parênteses, uma única vez ao seu final, obedecidas, quando for o caso, as prescrições da alínea ``c''.       (Redação dada pela Lei Complementar nº 107, de 26.4.2001)

\parun O termo ‘dispositivo’ mencionado nesta Lei refere-se a artigos, parágrafos, incisos, alíneas ou itens.     (Inciso incluído pela Lei Complementar nº 107, de 26.4.2001)

\capitulo{DA CONSOLIDAÇÃO DAS LEIS E OUTROS ATOS NORMATIVOS}

\secao{Da Consolidação das Leis}

\art As leis federais serão reunidas em codificações e consolidações, integradas por volumes contendo matérias conexas ou afins, constituindo em seu todo a Consolidação da Legislação Federal.   (Redação dada pela Lei Complementar nº 107, de 26.4.2001)

\so A consolidação consistirá na integração de todas as leis pertinentes a determinada matéria num único diploma legal, revogando-se formalmente as leis incorporadas à consolidação, sem modificação do alcance nem interrupção da força normativa dos dispositivos consolidados.     (Inciso incluído pela Lei Complementar nº 107, de 26.4.2001)

\so Preservando-se o conteúdo normativo original dos dispositivos consolidados, poderão ser feitas as seguintes alterações nos projetos de lei de consolidação:     (Inciso incluído pela Lei Complementar nº 107, de 26.4.2001)

\inc introdução de novas divisões do texto legal base;     (Inciso incluído pela Lei Complementar nº 107, de 26.4.2001)

\inc diferente colocação e numeração dos artigos consolidados;      (Inciso incluído pela Lei Complementar nº 107, de 26.4.2001)

\inc fusão de disposições repetitivas ou de valor normativo idêntico;    (Inciso incluído pela Lei Complementar nº 107, de 26.4.2001)

\inc atualização da denominação de órgãos e entidades da administração pública;     (Inciso incluído pela Lei Complementar nº 107, de 26.4.2001)

\inc atualização de termos antiquados e modos de escrita ultrapassados;    (Inciso incluído pela Lei Complementar nº 107, de 26.4.2001)

\inc atualização do valor de penas pecuniárias, com base em indexação padrão;       (Inciso incluído pela Lei Complementar nº 107, de 26.4.2001)

\inc eliminação de ambigüidades decorrentes do mau uso do vernáculo;       (Inciso incluído pela Lei Complementar nº 107, de 26.4.2001)

\inc homogeneização terminológica do texto;     (Inciso incluído pela Lei Complementar nº 107, de 26.4.2001)

\inc supressão de dispositivos declarados inconstitucionais pelo Supremo Tribunal Federal, observada, no que couber, a suspensão pelo Senado Federal de execução de dispositivos, na forma do art. 52, X, da Constituição Federal;      (Inciso incluído pela Lei Complementar nº 107, de 26.4.2001)

\inc indicação de dispositivos não recepcionados pela Constituição Federal;     (Inciso incluído pela Lei Complementar nº 107, de 26.4.2001)

\inc declaração expressa de revogação de dispositivos implicitamente revogados por leis posteriores.     (Inciso incluído pela Lei Complementar nº 107, de 26.4.2001)

\so As providências a que se referem os incisos IX, X e XI do § 2o deverão ser expressa e fundadamente justificadas, com indicação precisa das fontes de informação que lhes serviram de base.   (Inciso incluído pela Lei Complementar nº 107, de 26.4.2001)

\art Para a consolidação de que trata o art. 13 serão observados os seguintes procedimentos:     (Redação dada pela Lei Complementar nº 107, de 26.4.2001)

\inc O Poder Executivo ou o Poder Legislativo procederá ao levantamento da legislação federal em vigor e formulará projeto de lei de consolidação de normas que tratem da mesma matéria ou de assuntos a ela vinculados, com a indicação precisa dos diplomas legais expressa ou implicitamente revogados;      (Redação dada pela Lei Complementar nº 107, de 26.4.2001)

\inc a apreciação dos projetos de lei de consolidação pelo Poder Legislativo será feita na forma do Regimento Interno de cada uma de suas Casas, em procedimento simplificado, visando a dar celeridade aos trabalhos;    (Redação dada pela Lei Complementar nº 107, de 26.4.2001)

\inc revogado.   (Redação dada pela Lei Complementar nº 107, de 26.4.2001)

\so Não serão objeto de consolidação as medidas provisórias ainda não convertidas em lei.     (Inciso incluído pela Lei Complementar nº 107, de 26.4.2001)

\so A Mesa Diretora do Congresso Nacional, de qualquer de suas Casas e qualquer membro ou Comissão da Câmara dos Deputados, do Senado Federal ou do Congresso Nacional poderá formular projeto de lei de consolidação.    (Inciso incluído pela Lei Complementar nº 107, de 26.4.2001)

\so Observado o disposto no inciso II do caput, será também admitido projeto de lei de consolidação destinado exclusivamente à:    (Inciso incluído pela Lei Complementar nº 107, de 26.4.2001)

\inc declaração de revogação de leis e dispositivos implicitamente revogados ou cuja eficácia ou validade encontre-se completamente prejudicada;    (Inciso incluído pela Lei Complementar nº 107, de 26.4.2001)

\inc inclusão de dispositivos ou diplomas esparsos em leis preexistentes, revogando-se as disposições assim consolidadas nos mesmos termos do § 1o do art. 13.     (Inciso incluído pela Lei Complementar nº 107, de 26.4.2001)

\so (VETADO)     (Incluído pela Lei Complementar nº 107, de 26.4.2001)

\art Na primeira sessão legislativa de cada legislatura, a Mesa do Congresso Nacional promoverá a atualização da Consolidação das Leis Federais Brasileiras, incorporando às coletâneas que a integram as emendas constitucionais, leis, decretos legislativos e resoluções promulgadas durante a legislatura imediatamente anterior, ordenados e indexados sistematicamente.

\secao{Da Consolidação de Outros Atos Normativos}

\art Os órgãos diretamente subordinados à Presidência da República e os Ministérios, assim como as entidades da administração indireta, adotarão, em prazo estabelecido em decreto, as providências necessárias para, observado, no que couber, o procedimento a que se refere o art. 14, ser efetuada a triagem, o exame e a consolidação dos decretos de conteúdo normativo e geral e demais atos normativos inferiores em vigor, vinculados às respectivas áreas de competência, remetendo os textos consolidados à Presidência da República, que os examinará e reunirá em coletâneas, para posterior publicação.

\art O Poder Executivo, até cento e oitenta dias do início do primeiro ano do mandato presidencial, promoverá a atualização das coletâneas a que se refere o artigo anterior, incorporando aos textos que as integram os decretos e atos de conteúdo normativo e geral editados no último quadriênio.

\capitulo{DISPOSIÇÕES FINAIS}

\art Eventual inexatidão formal de norma elaborada mediante processo legislativo regular não constitui escusa válida para o seu descumprimento.

\artX (VETADO)     (Incluído pela Lei Complementar nº 107, de 26.4.2001)

\art Esta Lei Complementar entra em vigor no prazo de noventa dias, a partir da data de sua publicação.

Brasília, 26 de fevereiro de 1998; 177º da Independência e 110º da República.


\end{document}